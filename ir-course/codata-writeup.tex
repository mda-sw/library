\documentstyle[palatino,11pt]{article}
\textheight=21truecm 
\textwidth=15truecm  
\begin{document}

\section*{Content-Based Information Retrieval: New Tools for Textual 
Data, New Problems for Image Data}

\bigskip
 
\

{\bf Fionn Murtagh}
 
Space Telescope -- European Coordinating Facility
 
European Southern Observatory, Karl-Schwarzschild-Str. 2
 
D-85748 Garching, Germany
 
{\tt fmurtagh@eso.org}
 
\bigskip
 
\bigskip
 


{\bf Abstract:}
Retrieval of data and 
information is something which every scientist and technologist
does many times over in each working day.  However there are still problems 
to be overcome before heterogeneous data can be integrated for the 
user in a seamless way.  We discuss three case-studies.  The first deals with
textual information retrieval
in a modern science environment, using a particular set of text segments
-- astronomical observing proposal abstracts.
Secondly, we discuss viable approaches to content-based image retrieval, using
accompanying textual information.
Finally we examine the role that multiresolution vision models can play in 
facilitating content-based image retrieval.  The immediate focus of this
third case-study is the
retrieval of information about objects, characterized by their morphological
shapes, from large image databases.



\section{Introduction}

The automation of text and image retrieval is not sufficiently 
standardized at present.  Yet many if not most of those involved in 
science and technology face the need to interrogate large data collections,
using content and context.  A number of facets of this problem will be 
discussed.  The application areas will be noted.  Some current solution 
techniques will be described.  

\section{Free-Text Information Retrieval}

Astronomical observation proposals are submitted for peer review at regular
time-intervals.  In the case of Hubble Space Telescope observations, the
abstracts are available for successful proposals.  These abstracts provide
convenient raw-material for description of the images which are taken later.
Associating  the abstracts and images will be described in section 3, below.  
Firstly we address the following issue: just as in the case of text 
segments in other fields (company reports, for financial appraisal purposes;
legal reports; etc.), can the text segments be used for other purposes also?
If so, what types of useful operations can be performed on them?

Three objectives of free text querying of 
observation proposal abstracts are envisaged:
 
\begin{enumerate}
 
\item As an aid to observation proposal assessors, and scheduling planners,
it is helpful to look for closely related proposals.
 
\item As an aid to the user who is preparing a proposal, it is useful to
check on what has been already achieved, and by whom, using general phrases
which express his or her envisaged research.
 
\item As a general approach to image retrieval from the HST image archive,
we may search through the proposal abstracts, which we hope express in a 
few natural language sentences what the associated images (observed later)
are meant to contain.
 
\end{enumerate}

An  HST observing proposal abstract is shown in the following.
 
\begin{footnotesize}
\begin{footnotesize}
\begin{verbatim}
Prop. Type:  GO                                                                
 
              GALAXIES & CLUSTERS   -- (  GAS & DUST  ) -- 
3840- CT - "THE ABUNDANCES AND TIME EVOLUTION OF CARBON, NITROGEN AND OXYGEN IN
               STAR-FORMING GALAXIES"          
    Continuation of Program Number  3840       
    Keywords :  
 
    Proposers: Evan D. Skillman (PI; University Of Minnesota), R.Dufour (Rice 
               University), D.Garnett (Space Telescope Science Institute), 
               M.Peimbert (Unam; Mexico), G.Shields (University Of Texas), 
               E.Terlevich (Royal Greenwich Observatory; United Kingdom), 
               R.Terlevich (Royal Greenwich Observatory; United Kingdom), 
               S.Torres-Peimbert (Unam; Mexico)   
 
    We propose to observe UV emission lines of carbon, nitrogen, and oxygen 
    from high-surface brightness extragalactic H II regions drawn from a sample
    of irregular and spiral galaxies having a large spread of known oxygen  
    abundance (2% solar to nearly solar). From the emission-line data we will 
    derive C/O and N/O abundance ratios for which systematic uncertainties -
    due to reddening corrections, temperature and density effects, and     
    mismatched aperture sizes (typical in IUE+optical studies) - are greatly 
    reduced. We will use the derived abundances to study the time evolution of
    C/O and N/O in nearby galaxies, and compare the results with those obtained
    from observations of stars in our own Galaxy. We will be able to test the 
    suggestion (from far-infrared observations of H II regions) that nitrogen 
    abundances derived from optical spectra are systematically in error by
    factors of two or more. We will also be able to measure the gas phase 
    abundance of silicon, allowing us to study Si depletion as a function of 
    metallicity. Our target sample size of 28 is sufficiently large to study 
    both trends in relative abundances and search for anomalous regions (for
    example, those affected by the presence of WR stellar winds). The order of
    magnitude increase in s/n over IUE will allow the measurement of C/O and
    C/N with the requisite accuracy for the first time.      
                                                                      
\end{verbatim}
\end{footnotesize}
\end{footnotesize}

As an experiment in constructing an astronomically knowledgeable search engine,
we took the IAU Thesaurus (Shobbrook and Shobbrook, 1993) to generate a list
of astronomical phrases.  No particular attention was paid to 
object names (e.g.\ globular cluster 47 Tuc); or to additional semantics 
related especially to data in numeric form (``wavelengths less than 7000 \AA'';
``distant''; etc.).  

We took the accepted observing proposals, for Hubble Space Telescope, for 
the first four observing 
cycles (1989--1993), and experimented with publicly available 
information retrieval software tools which support free text querying.  
About 1500 proposal abstracts were used.  Due to the control allowed the
user in supporting querying by phrase (``IRAS galax*'', ``spectral energy
distribution'', ``elliptical galax*'', ``spectral type'', ``absorption line''),
its ignoring dashes, and its case insensitivity, a 
publicly available Unix tool, {\it lq-text}, was 
used.  This software is available by anonymous ftp from
{\tt ftp.cs.toronto.edu}.  Problems which  remained included the limited
capabilities to handle such astronomical 
terms as ``A star'', ``B star'', etc. 

A set of 286 phrases was ultimately retained, each of which is present 
in one or more
proposal abstracts.  To assess the potential for such phrases to describe
the interrelationships between the proposal abstracts, a clustering was 
carried out on the 1434 proposals, crossed (presence/absence) by the 
286 phrases.  The Kohonen ``self-organizing feature map'' 
(Murtagh and Hern\'andez-Pajares, 1994) offers a 
simultaneous clustering and dimensionality reduction.  Characterizing the 
clusters found on a $8 \times 8$ regular grid was quite successful.  Major
categories of observations were obtained: ``stars'', ``planetary nebulae'',
``solar system'', etc.  
See Murtagh (1994) for further details of this particular study.

\section{Content-Based Image Retrieval}

It is not an easy matter  to characterize meaningfully, and for general 
purposes, the subject matter of images. One possible solution is to use 
textual 
information accompanying an image.

We constructed such a prototype for the HST archive, which around the time
of the CODATA'94 meeting contained about 50,000 2-dimensional and 
1-dimensional (spectral) images.  The set of available observing proposals
had grown to 2006, and these provided a convenient body of text to 
characterize the images.  A proposal was an advance statement of what the 
image was to contain, so a working hypothesis was that the image actually
was related to the description in the observing proposal.  

More than one image could be related to a proposal.  For instance, a 
snapshot survey of 
protoplanetary nebulae requested, and subsequently obtained, 100 different
images.  

In an initial prototyping phase we concentrated on Wide Field/Planetary Camera
(WF/PC) images, taken before HST's refurbishment in December 1993.  Each 
such image is of dimensions $1600 \times 1600$, and is 10.5 MB in size.  
However many images have a preview version.  This is a binned, and compressed
version (using a wavelet H-transform), and leads to images of dimensions 
approximately $800 \times 800$, of size 40 kB, which can be stored on-line.
When using such images, we further reduced their size and converted them to 
GIF format (in order to allow them to be viewed inline, when using a World-Wide
Web browser such as Mosaic).

About 4340 unique images 
(different coordinates of the sky were observed; but we 
allowed for exclusion of images of the same location on the sky with different
exposure times, or using different filters, simply to cut down on the 
processing involved in this prototype) were finally sought, and 
2362 preview images 
were found to be available.  Generally, non-existence of a preview
image means that the proprietary period of one year is still running.  

We used WAIS (Wide-Area Information Server) to index the observing proposal
text, having first created an HTML file from the proposal with a link to any
accompanying images.  Then, accessing WAIS via Mosaic provided a list of 
``hits'' resulting from a free-text query.  Each item in this ranked list
corresponded to a proposal.  Clicking on any such item provided the text of 
the proposal, and associated images which were displayed in an inline manner.

This approach is quite a feasible one for the quick and easy delivery of 
information.  It is flexible, in that the user's request is
expressed in free-text.  Figure 1 shows the result in the case of a particular
proposal.  The query term used was ``extragalactic distance scale''.  The
prototype version set up is at this time accessible at URL
{\tt http://ecf.hq.eso.org/$\sim$fmurtagh/hst-navigate.html}.  

\begin{figure}
\vskip 6cm
\caption{Text (proposal abstract) and inlined image(s) found when using a 
free-text search.  This illustrates 
one approach to content-based image retrieval.}
\end{figure}


\section{Image Retrieval through Image Inventory}

A more difficult way to carry out content-based image retrieval is 
through first analyzing the image, using appropriate pattern recognition
methods, and building up an inventory of the image's contents.  
Traditional approaches to object inventory have used thresholding.  An
enhanced approach to adaptive thresholding is used here which incorporates 
a vision model relating to the type of object which is sought as a priority.
This vision model is based on multiresolution analysis, and mathematical 
morphological operations.

Multiresolution analysis (Starck et al., 1994)
is motivated by the fact that the human visual system deals with visual
scenes at differing resolution scales.  It handles these resolution scales 
simultaneously.  Using wavelet transforms, or other approaches (Starck et al.,
1994), the first phase is to arrive at a set of 4 or 5 resolution scales
related to the original image.  Next we use the fact that astronomical images
are always noisy.  Hence we seek to find what is real and what is noise-related
in the set of multiresolution images.  We represent the images at each scale
by boolean versions, -- a one represents a pixel associated with a real 
object, and a zero represents a noise pixel.  Contiguous sets of 1-valued 
pixels demarcate astronomical images of interest.  This allows a range of 
quantitative properties of the objects, i.e.\ the pixels in a given contiguous
``island'', to be determined.  

A report is built up on the objects found.
Figures 2 and 3 show the first stages of this.  These results relate to the 
searching in a large number of images, for faint edge-on galaxies.  
Figure 2 shows the original image, where some objects are very faint indeed.
It is a WF/PC image from HST. 
This image was filtered to remove the spoiling effect of cosmic-ray hits --
a common problem for such CCD detectors when the Earth's atmosphere provides
little protection.  Figure 3 depicts the 
multiresolution-related boolean images.
Further processing involves performing mathematical morphology {\tt opening}
operations on Figure 3, to remove detector artifacts (long linear features)
and the many left-over effects of cosmic ray hits.  Then the remaining objects
are labeled, and quantitative characterizations made (peak-to-background
intensity, roundness, moments, ellipticity, etc.).  


\begin{figure}
\vskip 6cm
\caption{An image (WF/PC, from Hubble Space Telescope).  We seek to 
demarcate and then characterize the astronomical objects which are present.}
\end{figure}

\begin{figure}
\vskip 6cm
\caption{From the image shown in Figure 2, boolean images are created which
serve to specify  -- on varying resolution levels -- the object present.}
\end{figure}

The problem
of finding effective approaches which span varying image exposure levels is
not an easy one, and is being tackled empirically.  The multiresolution 
approach, in producing boolean images, allows Minkowski (mathematical
morphology) operators  such as {\tt openings} and {\tt closings} 
to be used on these
boolean images, such that the user's a priori knowledge of objects of interest
is availed of.  For example, objects which are too small to be of practical
use can be suppressed by a sequence of {\tt openings} ({\tt erosion} followed 
by {\tt dilation}),
and this often includes the left-over effects of cosmic ray hits.  The 
kernel or structuring element used in the multiresolution transform, and 
later in the use of Minkowski operators, are related to the types of objects
wanted, -- e.g.\ point-symmetric shapes of limited extent.  Convolving with 
appropriate kernels yields clearer peaks, better definition of regions and
greater ease of implementation of subsequent image analysis operations.


\section{Conclusion}

Section 1 reviewed objectives, and one implementation approach, for the 
handling of relatively homogeneous ``chunks'' of text.  Software tools such 
as WAIS have progressed significantly in recent years.  Such tools are
distributed, i.e.\ the user and the server can be geographically distant.
Recent versions support fields (e.g.\ author name, or date), and the 
association of files of different types (e.g.\ text and image).  We believe 
that there is some way to go, still, before such tools are as often used
as, say, a news reader or an image display utility on the user's machine.  
Such tools provide access to text or image information; but text analysis
provides not only secondary analyses, but also a different type of access --
not based on novelty, but rather on directed subject-driven search.

Section 2 described a prototype image retrieval system, based on natural
language query specification.  The aim was to express the image's contents
using associated text fragments which are conveniently available.  A 
challenging task which lies ahead is to bring about more linkages with 
online abstract servers, which comprehensively represent the published
astronomical literature.

Section 3 described on-going research in using idealized astronomical 
objects in order to catalog, or perform an inventory of, all objects in an
image.  The image databases of the future will be interrogated in natural
language, and will reveal their secrets in a way which is more natural than 
today's cumbersome access mechanisms.
 
\section*{Acknowledgements}

The work described in section 3 is in collaboration with E.J.A.M. Meurs,
Dublin Institute of Advanced Studies.  A similar study, involving globular
cluster systems in the neighborhood of elliptical galaxies, is ongoing with
W. Zeilinger, University of Vienna.  Section 3 is based on work with J.-L. 
Starck, CEA Paris,  and A. Bijaoui, Observatoire de la C\^ote d'Azur, Nice.

\section*{References}

\begin{enumerate}

\item F. Murtagh, ``Free text information retrieval: an assessment of 
publicly available Unix-based systems'', report (unpublished), Feb. 1994, 
12 pp.

\item F. Murtagh and M. Hern{\'a}ndez-Pajares, M., ``The Kohonen 
     self-organizing map method: an assessment'', {\it Journal of 
     Classification},  1994, in press.

\item R.M. Shobbrook and R.R. Shobbrook, {\it The Astronomy Thesaurus}, Version
1.1, IAU, 1993.

\item J.-L. Starck, A. Bijaoui and F. Murtagh, ``Multiresolution and 
astronomical image processing'', in H. Payne, D. Shaw and J. Hayes,
Eds., {\it ADASS'94 Astronomical Data Analysis Software and Systems},
ASP, San Francisco, 1994, forthcoming.

\end{enumerate}


\end{document}

\bye


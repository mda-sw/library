\documentstyle[palatino,11pt]{article}
\textheight=21truecm            % height of the text on the page
\textwidth=15truecm             % width of the text on the page
\begin{document}

\section*{Free Text Information Retrieval:}

\section*{An Assessment of Publicly Available Unix-Based Systems}

\bigskip

\bigskip

F. Murtagh, Feb. 1994

\bigskip

\bigskip

{\bf Summary:} Arising out of interest in approaches for content-based
image retrieval from the HST image archive, we prototyped free text
retrieval systems using accepted proposal abstracts (hence: what the 
later associated images were meant to deal with).  Other possible 
applications of such free text retrieval systems 
in  observatory operations are described.  We used 
widely available, publicly accessible, (Sun) Unix-based  software tools only.
We discuss these experiments here in terms of client-server implementation, 
support for phrases in queries, etc.  We also review the applicability of the
IAU Astronomical Thesaurus, and object name resolvers, in this context.  We 
conclude with a number of recommendations for future work.

\section{Introduction}

To support information retrieval in astronomy, we believe that it is 
advantageous if the retrieval engine is astronomically-knowledgeable.  
In other words, the user ought not to have to fight against the limitations
of the indexing or querying system.  Consider, for example, a retrieval 
system which rejects a partially numeric object name 
(e.g.\ 47 Tuc) on the grounds of supporting character-text only.

Organisationlly, we see this work as: (i) selecting important desiderata,
and (ii) finding tools which support these requirements.

As such desiderata, we have chosen the following issues: (i) Is there a 
comprehensive astronomical vocabulary available, which ought to be in 
large measure supported by a retrieval system?  We use the IAU Thesaurus,
towards this end.  (ii) The next phase will involve support of astronomical
object names: what tools are available to support free text querying based 
on object names?  (iii) A third phase will seek to go beyond that: to 
support additional semantic requests (e.g. not simply the term ``wavelength'',
but rather ``wavelengths less than 7000 
\AA''; fuzzy attributes such as ``distant'', etc.).


\section{Data: HST accepted observing proposal abstracts}

We took the accepted observing proposals, for HST, for the first
four cycles (1989--1993), and experimented with publicly available 
information retrieval software tools which support free text querying.  
Imposed keywords may be available as part of the proposal texts.  However
such imposed keyword systems are often either limited in scope, or are only
partially carried out. Free text deriving of keywords is a feasible 
alternative, which is considerably more user-friendly. 

A typical HST observing proposal abstract is shown in the following.

\begin{footnotesize}
\begin{verbatim}
     676                                             676propnum                
Prop. Type:  GO                                                                

              GALAXIES & CLUSTERS   -- (  GAS & DUST  ) -- 
3840- CT - "THE ABUNDANCES AND TIME EVOLUTION OF CARBON, NITROGEN AND OXYGEN IN
               STAR-FORMING GALAXIES"          
    Continuation of Program Number  3840       
    Keywords :  

    Proposers: Evan D. Skillman (PI; University Of Minnesota), R.Dufour (Rice 
               University), D.Garnett (Space Telescope Science Institute), 
               M.Peimbert (Unam; Mexico), G.Shields (University Of Texas), 
               E.Terlevich (Royal Greenwich Observatory; United Kingdom), 
               R.Terlevich (Royal Greenwich Observatory; United Kingdom), 
               S.Torres-Peimbert (Unam; Mexico)   

    We propose to observe UV emission lines of carbon, nitrogen, and oxygen 
    from high-surface brightness extragalactic H II regions drawn from a sample
    of irregular and spiral galaxies having a large spread of known oxygen  
    abundance (2% solar to nearly solar). From the emission-line data we will 
    derive C/O and N/O abundance ratios for which systematic uncertainties -
    due to reddening corrections, temperature and density effects, and     


    mismatched aperture sizes (typical in IUE+optical studies) - are greatly 
    reduced. We will use the derived abundances to study the time evolution of
    C/O and N/O in nearby galaxies, and compare the results with those obtained
    from observations of stars in our own Galaxy. We will be able to test the 
    suggestion (from far-infrared observations of H II regions) that nitrogen 
    abundances derived from optical spectra are systematically in error by
    factors of two or more. We will also be able to measure the gas phase 
    abundance of silicon, allowing us to study Si depletion as a function of 
    metallicity. Our target sample size of 28 is sufficiently large to study 
    both trends in relative abundances and search for anomalous regions (for
    example, those affected by the presence of WR stellar winds). The order of
    magnitude increase in s/n over IUE will allow the measurement of C/O and
    C/N with the requisite accuracy for the first time.      
                                                                      
\end{verbatim}
\end{footnotesize}

The proposal number, here, is 3840.  The number 676 is a sequence number which
we inserted, and used for certain experiments.  A number of line skips, 
new page indicators, etc. have been removed (to improve display) in this 
sample.  

Three objectives of free text querying of 
observation proposal abstracts are envisaged:

\begin{enumerate}

\item As an aid to observation proposal assessors, and scheduling planners,
it is helpful to look for closely related proposals.

\item As an aid to the user who is preparing a proposal, it is useful to
check on what has been already achieved, and by whom, using general phrases
which express his or her envisaged research.

\item As a general approach to image retrieval from the HST image archive,
we may search through the proposal abstracts, which we hope express in a 
few natural language sentences what the associated images (observed later)
are meant to contain.

\end{enumerate}

The 1434 HST proposal abstracts in the first four cycles of HST observing 
contained both GTO (Guaranteed Time Observer) and GO (Guest Observer) 
proposals.  About 400 propsal abstracts were duplicated in
the initial set of experiments (WAIS-related) carried out, so that we
dealt with about 1854 proposals in these initial experiments.  
Apart from a minor 
additional computational requirements, there was no difficulty with
such redundancy. 

Typically a proposal abstract consisted of from one to two 
dozen lines of free text, 
preceeded by title, principal investigators and co-investigators, 
sometimes keywords, a proposal number in 
one of about half a dozen line layout formats, extraneous character 
information such as
page feeds, occasional extraneous comments relating to changes in the 
proposal, occasional mispellings (``Rayleight''), sections containing 
short lists of acronyms, etc.  However the essential
proposal abstract was clear to a human reader, and we sought to use the
abstract information with only limited cleaning and reformatting.

In the first section below, detailing WAIS, all abstracts were in one ASCII 
file
with a line of dash separators between successive abstracts.  In the section
detailing {\it lq-text}, each abstract was put in an individual 
ASCII file.

Unix tools such as {\it grep} are not usable, on the grounds of computational
requirements.  Inverted files are necessary, and the tools described here 
use such data structures to facilitate subsequent free text searching.


\section{Free text retrieval, using a client-server tool}

WAIS (Fullton, 1993) 
%permits the user's client software to collect together 
%the necessary elements of the query, and to send them via network to the
%database host, where the server software carries out the search and returns
%the results.  In such a situation the server software has already been used
%to construct an inverted file of the free text to be searched through.
%WAIS 
is a networked information retrieval system, using TCP/IP
to connect client applications to information servers.  The Z39.50 query 
protocol is used to communicate between clients and servers.

We began with WAIS, given the usefulness of this client-server approach to 
information retrieval, but we found difficulties which made us subsequently
look for alternative tools.  The version of WAIS used is important, given that
various versions are in use: we used freeWAIS version 0.202 for indexing
and WAIS version 8 beta 5 for querying.  It must be
stressed that the default mode of operation of WAIS was used.  
Other versions of WAIS could have been considered, as well as changes 
or enhancements to the code.

WAIS allows any number of a large range of file formats to be indexed.  We
chose to have dash separating lines between proposal abstracts, which were
all stored in one large file.  Case (lower, upper) is ignored when using
WAIS. This property is availed of also in the case of other retrieval tools 
discussed below.

We found that single-word queries were
necessary, rather than phrases (this necessity will be discussed below).  
This had the unwelcome by-product that a query such as ``X ray'' would 
seek matches associated with ``X'' and with ``ray''; and matches with the 
latter word would yield hits with the author name ``Ray'' of which there
were a number in the HST abstract proposal collection.  To bypass this
problem, we used
a version of the abstract proposal from which we removed all personal names
of principal investigators and co-investigators. 
In a similar way, a number of occurrences of header-type information,
including details of certain acronyms used, had to be deleted from the 
HST abstract proposals file before indexing.  If WAIS had allowed us to 
use strict phrase queries, this problem could have been mitigated.

In what sense 
is phrase support for queries lacking in WAIS?  If a query
consists of the terms ``absolute temperature scale'', one will get hits 
related to ``absolute magnitude'' or to ``distance scale''. (In the version
of WAIS used, putting the search string in quotation marks did not alter
the fact that WAIS provided the best set of hits, if necessary by 
breaking up the search string.)  WAIS provides
best-matching results, if necessary by being very flexible with the 
arrangement of the words used in the user's query.  But astronomy is a 
``phrase-rich'' domain: phrases such as those mentioned must be kept in their
given form.  Faced with the dangers involved in WAIS's treatment of phrases,
we took the decision to only allow single-word queries.  At least in doing
that, we thought, we could impose greater (semantic, astronomical) control
over WAIS.

The IAU Thesaurus (Shobbrook and Shobbrook, 1993) is a 
(hierarchically-structured) list of 
astronomically-appropriate words and phrases.  We broke up all pharses in the
Thesaurus
into single words, and removed replications.  We deleted terms beginning
with numbers (e.g. 13, from ``13 cm line'') since these would be confused
with HST abstract proposal numbers.  We added a list of chemical and 
atomic element
names, and names of planets, constellations, etc.  From our list of 
words, we deleted trailing plurals (``s'', ``ies'', ``ae'') and replaced 
these with a wild-card symbol (*), since WAIS supports wild-cards in 
querying.  

We then set up a script to run a WAIS query associated with each word in our
list: in all, 1715 queries.  This took about 80 minutes CPU time on a 
SPARCstation 10.  A number of programs were then written to create and
process the resulting proposal abstract versus index term association matrix. 

Hits with all WAIS-allowed scores (1000 downwards) were retained. 
In all queries, the maximum number of proposal abstracts returned ceilinged
at 264.  This was related to the particular scoring mechanism which gives 
additional weight to multiple occurrences of terms, and 
normalizes with respect to the best retrieval in the database.

%The index term versus proposal abstract dependency matrix is shown in 
%Fig. y.  We recall that there were a few hundred repeated occurrences of 
%abstracts here.  
We experimented with a number of ways to summarize this
information: row and column permuting, as successfully used by Packer (1989)
for citation analysis; correspondence analysis and principal components
analysis; and the Kohonen neural network method, a simultaneous 
clustering and non-linear dimensionality-reduction method.  We obtained
good results, in reordering rows and columns, when we used rare index terms
(less than or equal to 20 associated proposal abstracts).  For clustering,
we found difficulty in characterizing the clusters found.  
This was related to the lack of semantic content in isolated
words (``absolute'', ``active'', ``apparent'', ``brightness'', etc.).  We
concluded that the strict treatment of phrases was a sine qua non for
taking astronomical semantics into account. 


\section{Free text retrieval, supporting phrases, using lq-text}

{\it Lq-text}, due to Liam Quin, 
 is available by anonymous ftp from
ftp.cs.toronto.edu.  We used version 1.13 (although individual
commands had version numbers preceding this).  {\it Lq-text} is a text
information retrieval utility, using indexing,  which 
supports phrases and keyword-in-context.
It does not support the Z39.50 or related protocols, and is not 
immediately interfaceable to WAIS.  It is currently  still under development.

Advantages of {\it lq-text} include: phrase retrieval is supported,
including phrases which are broken by ends of lines; dashes are 
ignored (``Lyman-alpha'', ``Lyman alpha''); and in the default 
mode, case is ignored.  Disadvantages include: in the default
setting, words are between 3 and 18 characters in length (thus
``UV'' is ignored); boolean queries are supported but 
awkwardly, by ranking and intersecting; and strings which consist
entirely of numeric data, or start with numeric characters, are
excluded. 

Following indexing, the indexing directory can be specified explicitely or
with an environment variable. 
%The indexing is carried out. The indexes are directed into 
%the directory specified in environment variable LQTEXTDIR.  
%Specification of directory containing indexes is necessary.  
In the following
the ordinary Unix system prompt is {\tt command>}, and 
{\it lq-text}'s response is shown in footnote size font.

%\begin{verbatim}
%
%command> setenv LQTEXTDIR lq-text-dir
%
%\end{verbatim}

Query phrase support (qualifier ``v'' implies verbose mode):

\begin{verbatim}
command> lqphrase -v "IRAS galaxies"

\end{verbatim}

\begin{footnotesize}
\begin{verbatim}
lqphrase: lqtext directory "/diskc/fmurtagh/lq-text-dir"
Word IRAS --> Iras, 32 matches
Word galaxies --> galaxies, 1981 matches
3       38      1   3   hstprop3913
\end{verbatim}
\end{footnotesize}

Although there were 32 matches with the word ``IRAS'', and 1981 with
``galaxies'', only in document ``hstprop3913'' did these come together as 
a phrase.  We are not interested here in the remaining numeric information 
returned, relating to offsets in the text of the word found. 
A KWIC (keyword in context) option follows.

\begin{verbatim}
command> lqkwik `lqphrase "IRAS galaxies"`
\end{verbatim}

\begin{footnotesize}
\begin{verbatim}
91) that ultraluminous   IRAS galaxies are      : hstprop3913
\end{verbatim}
\end{footnotesize}

Another support option is to show more of the context of what
has been found.  A screen with up to 6 (default) lines before 
and after the phrase or word, for each hit, is presented using
command ``lqshow''.

\begin{verbatim}

command> lqshow `lqphrase "IRAS galaxies"`

\end{verbatim}

\begin{footnotesize}

\begin{verbatim}

Block 3/Word 38 in document: hstprop3913

    the time dependence of the merger rate for massive galaxies. This will put

    on a firmer observational footing the present theoretical estimates of how

    many ellipticals were made by mergers. (2) We plan to pursue a suggestion

    by Kormendy, Sanders & Cowie (1991) that ultraluminous IRAS galaxies are

    local analogs of protogalaxies. The proposed objects have precisely the

    properties expected of elliptical galaxies forming by dissipative collapse.

    They contain intense starbursts that are shrouded by dust. Since even the

    oldest giant ellipticals have solar or higher metallicities, all giant


\end{verbatim}

\end{footnotesize}

Handling of singulars and plurals often worked well.
We did find some limitations on stemming.  For instance ``protogalaxies'' 
apparently truncated the ``s'', and provided hits corresponding to this 
plural only; whereas ``protogalaxy'' caught both singulars and plurals.

Multiple phrase support is illustrated in the following.  We get information
on the number of hits associated with the individual words in the phrase.
The list of documents, containing the phrase, has repititions when there 
are multiple occurrences of the phrase.

\begin{verbatim}

command>  lqphrase -v "IRAS starburst galaxies" "spectral energy distri
butions" "massive galaxies" "elliptical galaxies" "protogalaxies"

\end{verbatim}

\begin{footnotesize}

\begin{verbatim}

lqphrase: lqtext directory "/diskc/fmurtagh/lq-text-dir"

Word IRAS --> Iras, 32 matches
Word starburst --> starburst, 113 matches
Word galaxies --> galaxies, 1981 matches
1       1       3   99  hstprop3913
2       3       1   3   hstprop3913

Word spectral --> spectral, 272 matches
Word energy --> energy, 131 matches
Word distributions --> distributions, 282 matches
3       51      1   0   hstprop1045
4       2       15  64  hstprop2607
2       17      1   0   hstprop2719
3       7       1   0   hstprop3543
2       15      1   0   hstprop3913
3       18      1   0   hstprop4072
3       18      1   0   hstprop4142
3       19      95  96  hstprop4311
8       15      97  64  hstprop4581
6       22      1   0   hstprop4842
1       28      1   0   hstprop5395
3       60      1   0   hstpropFS23

Word massive --> massive, 168 matches
Word galaxies --> galaxies, 1981 matches
3       11      1   0   hstprop3913
2       0       1   0   hstprop5447

Word elliptical --> elliptical, 138 matches
Word galaxies --> galaxies, 1981 matches
1       3       3   98  hstprop1057
0       15      3   98  hstprop1114
0       16      1   2   hstprop1242
1       19      1   0   hstprop1248
2       35      9   64  hstprop1248
0       10      3   98  hstprop2295
0       13      1   2   hstprop2405
1       7       2   66  hstprop2405
2       16      1   0   hstprop2405
2       30      1   0   hstprop2405
5       4       1   0   hstprop2405
1       2       3   98  hstprop2607
2       50      1   0   hstprop2607
1       3       4   98  hstprop2719
1       3       3   98  hstprop3225
1       2       3   98  hstprop3265
0       16      1   130 hstprop3333
1       1       3   98  hstprop3448
2       27      1   0   hstprop3448
1       18      2   66  hstprop3545
1       2       3   98  hstprop3647
3       5       3   96  hstprop3647
2       27      1   0   hstprop3657
3       0       1   0   hstprop3657
3       52      3   96  hstprop3913
3       0       3   96  hstprop4085
1       2       3   98  hstprop4205
9       22      3   96  hstprop4366
6       1       3   96  hstprop4644
1       5       3   98  hstprop4838
5       10      3   96  hstprop4903
5       10      3   96  hstprop4904
6       1       3   96  hstprop4905
2       18      17  64  hstprop5396
2       32      1   0   hstprop5416
2       13      1   0   hstprop5430
3       43      15  64  hstprop5436
2       38      18  96  hstprop5446
4       15      3   96  hstprop5465
4       4       1   0   hstprop5496
0       6       127 66  hstpropFC03
1       4       2   66  hstpropFC03
2       14      1   0   hstpropFC03
2       4       3   96  hstpropWC02
0       7       3   98  hstpropWC08
1       9       3   96  hstpropWC08
1       7       16  96  hstpropWD02

Word protogalaxies --> protogalaxies, 9 matches
4       4       1   1   hstprop1233
3       54      1   1   hstprop3181
3       54      1   1   hstprop3331
3       43      3   97  hstprop3913

\end{verbatim}

\end{footnotesize}

Thus, in the last segment of what was retrieved (as shown just above), 
``protogalaxies'' was found 9 times, but 4 
documents were affected in all.

{\it Lq-text} offers good support for phrases.  
%We see no immediate
%need to constrain these phrases to come from the IAU Thesaurus.  
On the 
other hand, {\it lq-text} does not offer a basis for the support of 
astronomical object naming
conventions (cf.\ lack of support for numeric-starting, or fully numeric,
strings).  
%It is also a pity that it is not implemented in
%a client-server fashion, nor does it aim to be.

\section{Integrating the IAU Thesaurus with lq-text as retrieval engine}

The availability of a controlled vocabulary is advantageous for a user
who has not carried out querying before.  We used the IAU Thesaurus for 
this purpose.  The IAU Thesaurus is ``phrase rich'': two- and three-word
phrases are very common (``spectral types'', ``absorption lines'', etc.).
Thus {\it lq-text} is an appropriate retrieval system, for such phrases.

Potentially important 
properties  of the IAU Thesaurus were more difficult to use.
Use of the hierarchical structure of the words and phrases in the Thesaurus 
was eschewed.  Words or phrases at all hierarchical levels in the Thesaurus
were used on an equal footing.  This lead to redundancies: e.g. ``abell 
clusters'' appear under ``clusters'' and under ``clusters of galaxies''.  
Such redundancies were removed, leaving a total of 1427 phrases retained. 
(The original IAU Thesaurus list had contained 5356 lines, many dozen of
which were blank, separating lines.)

A simple script was used to run a query of the HST abstract data collection
corresponding to each of these phrases.  Thus {\it lq-text} was invoked 
1427 times.  The computational requirement for this was about one minute on
a SPARCstation 1.  

An additional problem was posed by the requirement of {\it lq-text}
that a word be three characters or greater in length.  This implied that
a phrase like ``Am star'' or ``Ap star'' both collapsed to 
``star''.  Under these
circumstances there was considerable redundancy in the queries.  To counter
this, the set of queries was liberally edited by hand to remove some of 
these phrases.  A few dozen phrases were deleted, for this reason.  
{\it Lq-text} also restricts usage of words beginning with numeric
characters (e.g. problematic for object names such as 47 Tuc).

The number of phrases which were found to be present one or more times in
the HST proposal abstract collection was 286.  The dependency array, crossing
1434 proposal abstracts with 286 phrases, is shown in Fig. 1.  There remain
redundant phrases: ``star'' or ``galaxy'' still appear a number of times.  
Thus, the vertical axis of Fig. 1 has a number of horizontal rows associated
with the word ``star''.  Such redundancy is not a problem, and causes at
worst an unnoticeable amount of computational overhead. 

The dependencies shown in Fig. 1 can -- in principle -- be used to provide
clusters of associated propsal abstracts, or to allow proposal abstracts to 
be located in a factor space (Ossorio, 1966) or in a space of dimensionality 
lower than 286 using latent semantic axes (Deerwester et al., 1990).  But
how feasible is this?  We checked out the clustering tendencies of the 
data shown in Fig. 1, and simultaneously the feasibility of projecting either
proposal abstracts, or index term phrases, non-linearly into a two-dimensional
space.  The Kohonen ``self-organizing'' feature map iteratively finds a 
grid structure of pseudo-proposal abstracts (286-dimensional vectors, which
will represent a cluster of proposal abstracts).  Using a regular $8 \times 8$ 
grid, we found the following numbers of proposal abstracts associated 
with each 
``bin'':

\begin{verbatim}

    37     0   241    44    41    11     1    70
     0     0    57     0     0     0     1    11
    26     0    52     3     8    73     0     2
    21     0    26     2     1     4     0    29
     8     0     9     0     2    15     0     2
    94     0    43     3     5    19     0    33
     6     0     7     0     0     0     4    10
   124    10    40     0    62     4    18   155

\end{verbatim}

Other grid dimensions could be used.  We see from the above, however, that
the largest cluster returned had 241 associated proposal abstracts.

To characterize the 64 ($= 8 \times 8$) ``bins'' above, we looked at the 
index term phrases which were strongly associated with each ``bin''.  
Technically, this procedure is as follows.  Each proposal abstracts is a
286-valued binary (0 or 1) vector.  Each ``bin'' vector, or pseudo-proposal
abstract, is initially a 286-valued vector with uniformly distributed random
values between 0 and 1.  The Kohonen procedure makes these ``bin'' vectors
accurately reflect an associated cluster of proposal abstracts.  When this
was done, we simply looked at each of the 286 values associated with a 
``bin'' vector, and a value greater than 0.5 in value $j$ ($1 \leq j \leq 
286$) indicated strong association between index term phrase $j$ and this 
``bin'' vector. 

The $8 \times 8$ grid was characterizable as follows:


\begin{verbatim}

---    ---    stars    stars     plan.     ---     solar    solar
                                 nebul.            syst.    syst.

---    ---    stars     ---       ---      ---     solar    solar
                                                   syst.    syst.

gal.   ---   astro-    astro-     solar   solar     ---      gal. 
              phys.     phys.     syst.   syst. 

gal.   ---   astro-    astro-     solar   solar     ---      QSOs
              phys.     phys.     syst.   syst.

gal./  ---   astro.     ---        ---    atrom.    ---      ---
QSOs          obs.

gal./  ---    obs.      obs.      motion  motion    ---     photom.
QSOs

QSOs   ---   QSOs       ---        ---     ---      gal.    gal.
                                                    clust.  clust.

QSOs   QSOs  QSOs       ---       QSOs     ---      gal.    gal.
                                                    clust.  clust.

\end{verbatim}

Here, ``gal.'' is ``galaxies'', ``plan. nebul.'' is ``planetary nebulae'',
``astrom.'' is ``astrometry'', ``gal. clust.'' is ``galaxy clusters'',
``astrophys.'' is ``astrophysics'', ``astro. obs.'' is ``astronomical 
observations'' and ``obs.'' is ``observations''.  With the exception 
of the latter three categories (which appear somewhat amorphous: we did
not pursue them further), most others are fairly clear.  
Furthermore, in general, they appear at contiguous locations in the
above diagram: note ``solar syst.'' or ``stars'', for example.  Allowing
for wrap-around (not explicitly catered for in our algorithmic 
implementation of the Kohonen method), the QSO-related ``bins'' are  also 
fairly adjacent to one another, compared to any other category.

This 2-d map of abstract proposals suffices for our purposes: the 
interrelationships and interdependencies between proposals and -- more
importantly --  between the index terms derived from the IAU Thesaurus
have a reasonable degree of internal consistency and coherence.  This
permits us to conclude that the IAU Thesaurus, used in this way, provides
at least an initial, suitable  controlled vocabulary for query-support
with this data.


\section{Other publicly available retrieval engines}

\begin{enumerate}

\item As already mentioned, {\it grep} 
and other related Unix tools are not feasible
on computational grounds.  The prior construction of an inverted file is a 
sine qua non for feasible free text information retrieval.

\item Qt (``Query Text'') is available from ftp.uu.net in 
usenet/comp.sources.unix/volume27.  It does not have phrase support 
(indirect possibilities do exist, through use of a permuted index), nor is
it a  client-server system.  It is effective for constructing and
handling an inverted file list; and boolean queries.  It is also 
well-documented, and easy to use.

\item Frakes and Baeza-Yates (1992) provide a comprehensive introduction to 
algorithmic information retrieval.  Accompanying C source code programs 
are available from ftp.vt.edu lexical analysis and handling stop lists,
stemming, thesaurus construction, boolean operations, and so on.  

\item Essence (available at ftp.cs.colorado.edu:/pub/cs/distribs/essence)
is a software tool, which produces WAIS-compatible indexes, but which 
addresses better the following issue: what type of information should one
derive from files which are not ASCII text files.  
Examples include binary, archive,
C and C header, directory, object, TeX, etc., files.  Ordinarily a user
will have files of these various types at hand. 



\end{enumerate}


\section{Conclusions}

We examined the client-server WAIS retrieval tool, and found certain
difficulties in its support of queries consisting of phrases; and 
its outputing of the best scoring hits was less useful for our purposes
than simply outputing all possible occurrences of the query terms.  
Needless to say, these conclusions are
based on the use of a particular version of WAIS.  

We examined the retrieval system, {\it lq-text}, and found it adequate
for our purposes in its support of phrases.  It is a stand-alone Unix
utility, and unless wedded to a system like WAIS, it does not run in a
client-server mode.  Again the version used by us should be taken into
account.  We have shown how a controlled vocabulary such as that provided
by the IAU Thesaurus can be integrated with the {\it lq-text} retrieval engine.
In fact, a user's desired observing program can easily be written in the 
form of a proposal abstract; this can be indexed by {\it  lq-text}, and 
all phrases retained from the IAU Thesaurus can be found in this text;
in the framework of our experiments, above, this would provide us with a 
286-valued binary vector corresponding to the user's proposed research; 
this vector can then be matched against all vectors associated with 
the 1434 accepted HST proposal abstracts in order to provide the user with 
a number of sufficiently similar abstracts. 

All in all, the IAU Thesaurus provides an adequate controlled vocabulary.
Further work will investigate the hierarchical structure of the Thesaurus,
and how that may be used to provide more semantic information in this 
integration of Thesaurus and retrieval engine. 

A difficulty in the use of {\it lq-text} 
arose in regard to supported word length (between
3 and 18 characters, which can be altered in the code).  Astronomical 
abbreviations are often of one or two characters.  Rather than altering
{\it lq-text} to allow for such abbreviations and short astronomical 
names, we see a need for a more thorough strategy in regard to support
of astronomical object names.  These should be processed separately.
A system such as SIMBAD provides a comprehensive name resolver, and 
also a hierarchical object classification scheme.  The first major problem
facing us
is to automatically extract all astronomical object names, or classes
of object, from free text.  Once this is done, a remote call to SIMBAD 
can provide synonym and classification information.

\section*{References}

\begin{enumerate}

\item J. Fullton, ``WAIS'', in A. Heck and F. Murtagh, Eds., {\it Intelligent
Information Retrieval: The Case of Astronomy and Related Space Sciences},
Kluwer, Dordrecht, 1993, pp. 113--118.

\item R.M. Shobbrook and R.R. Shobbrook, {\it The Astronomy Thesaurus}, Version
1.1, IAU, 1993.

\item P.G. Ossorio, ``Classification space: a multivariate procedure for
automatic document indexing and retrieval'', {\it Multivariate Behavioral
Research}, {\bf 1}, 479--524, 1966.

\item 
S. Deerwester, S.T. Dumais, G.W. Furnas, T.K. Landauer and R. Harshman,
``Indexing by latent semantic indexing'', {\it Journal of the American 
Society of Information Science}, {\bf 41}, 391--407, 1990.

\item C.V. Packer, ``Applying row-column permutations to matrix 
representations of large citation networks'', {\it Information Processing and
Management}, {\bf 25}, 307--314, 1989.

\item W.B. Frakes and R. Baeza-Yates, Eds., {\it Information Retrieval: Data 
Structures and Algorithms}, Prentice-Hall, 1992.

\end{enumerate}



\end{document}

\bye

